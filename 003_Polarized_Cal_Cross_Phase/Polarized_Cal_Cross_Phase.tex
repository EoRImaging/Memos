\documentclass[a4paper,11pt]{article}
\usepackage{amsmath, amsthm, amssymb, commath, graphicx, bbold, endnotes, graphicx, subfigure, multirow, setspace}
\usepackage[font=footnotesize]{caption}
\usepackage[top=1.25in, bottom=1.25in, left=1.25in, right=1.25in]{geometry}
\linespread{1}
\setlength{\parindent}{30pt}

\title{Cross Phase Calculation Memo}
\author{Ruby Byrne}
\date{April 2018}

\begin{document}

\maketitle


\section{Background}

One degree of freedom in polarized calibration is the overall phase offset between the $x$- and $y$-dipoles. This phase offset, called $\phi$ for the purposes of this memo, corresponds to a mixing between the Stokes U and V modes. As we do not expect sky emission to be circularly polarized, we can constrain $\phi$ by minimizing Stokes V emission. 

We use pseudo-Stokes parameters to simplify the calculation, allowing us to work in visibility space without gridding. Pseudo-Stokes parameters are calculated from the visibilities as follows:

\begin{equation}
\begin{bmatrix}
	I^{\text{pseudo}} \\
	Q^{\text{pseudo}} \\
	U^{\text{pseudo}} \\
	V^{\text{pseudo}} \\
\end{bmatrix}
=
\begin{bmatrix}
	1 & 1 & 0 & 0 \\
	1 & -1 & 0 & 0 \\
	0 & 0 & 1 & 1 \\
	0 & 0 & i & -i \\
\end{bmatrix}
\begin{bmatrix}
	xx^* \\
	yy^* \\
	xy^* \\
	yx^* \\
\end{bmatrix}
\end{equation}

It is clear that the transformation $x \rightarrow x e^{-i \phi/2}$ and $y \rightarrow y e^{i \phi/2}$ leaves pseudo-I and pseudo-Q unchanged, but affects the pseudo-U and pseudo-V parameters.


\section{Calculating the Cross Phase}

To calculate $\phi$, we minimize the quantity $\sum_n |V^{\text{pseudo}}_n|^2$ where $n$ denotes the visibility index. We write the visibilities in terms of an amplitude and phase, such that $(xy^*)_n = A_n e^{i\delta_n} \rightarrow A_n e^{i\delta_n - i\phi}$ and $(yx^*)_n = B_n e^{i\gamma_n} \rightarrow B_n e^{i\gamma_n + i\phi}$. Now 
\begin{equation}
V^{\text{pseudo}}_n = A_n e^{i(\delta_n - \phi + \pi/2)} + B_n e^{i(\gamma_n + \phi - \pi/2)}
\end{equation}
It follows that
\begin{equation}
|V^{\text{pseudo}}_n|^2 = A_n^2 + B_n^2 - 2 A_n B_n \left[ \cos(\delta_n - \gamma_n) \cos(2 \phi) + \sin(\delta_n - \gamma_n) \sin(2 \phi) \right]
\end{equation}
Now to find $\phi$ we set $\frac{\partial}{\partial \phi} \sum_n |V^{\text{pseudo}}_n|^2 = 0$. This gives
\begin{equation}
\tan(2 \phi) = \frac{\sum_n A_n B_n \sin(\delta_n - \gamma_n)}{\sum_n A_n B_n \cos(\delta_n - \gamma_n)}
\end{equation}
Rewriting this in terms of the visibilities, we get
\begin{equation}
\tan(2 \phi) = \frac{\sum_n \text{Im} \left[ (xy^*)_n (yx^*)_n{}^* \right]}{\sum_n \text{Re} \left[ (xy^*)_n (yx^*)_n{}^* \right]}
\end{equation}


\section{Implementation in FHD}

This calculation is implemented in FHD in the full-pol branch with the function \linebreak[4] \texttt{vis\_calibrate\_crosspol\_phase}. 

It is important to note that FHD convention stores visibilities in the \texttt{vis\_ptr} object, where counterintuitively \texttt{vis\_ptr[0]} corresponds to $yy^*$, \texttt{vis\_ptr[1]} corresponds to $xx^*$, \texttt{vis\_ptr[2]} corresponds to $yx^*$, and \texttt{vis\_ptr[3]} corresponds to $xy^*$.

The calculated value of $\phi$ is stored in the \texttt{cal.cross\_phase} object and the correct transformation is applied to the antenna gains. Since the raw visibilities are divided by the gains, the transformation $x \rightarrow x e^{-i \phi/2}$ is equivalent to $g_x \rightarrow g_x e^{i \phi/2}$ and $y \rightarrow y e^{i \phi/2}$ is equivalent to $g_y \rightarrow g_y e^{-i \phi/2}$. Here $g_x$ is the $x$-dipole gain (\texttt{cal.gain[1]}) and $g_y$ is the $y$-dipole gain (\texttt{cal.gain[0]}).

\end{document}